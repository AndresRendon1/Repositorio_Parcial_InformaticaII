\documentclass{article}
\usepackage[utf8]{inputenc}
\usepackage[spanish]{babel}
\usepackage{listings}
\usepackage{graphicx}
\graphicspath{ {images/} }
\usepackage{cite}

\begin{document}

\begin{titlepage}
    \begin{center}
        \vspace*{1cm}
            
        \Huge
        \textbf{Parcial #1 Info2}
            
        \vspace{0.5cm}
        \LARGE
            
        \vspace{1.5cm}
            
        \textbf{Integrantes:}
        \\
        \vspace{1.5cm}
        \textbf{Daniel Andres Agudelo Garcia}
        \\
        \textbf{Andres Felipe Rendon Villada} \\
        \textbf{Esteban Felipe Guiza Piñeros}
        
            
        \vfill
            
        \vspace{0.8cm}
            
        \Large
        Despartamento de Ingeniería Electrónica y Telecomunicaciones\\
        Universidad de Antioquia\\
        Medellín\\
        Marzo de 2021
            
    \end{center}
\end{titlepage}

\tableofcontents

\vspace{13cm}

\section{Introduccion: Analisis del problema}
Obervamos la problematica a tener en cuenta, en la cual planteamos diferentes soluciones y elegimos la mas optima, pensando en un desarrollo ideal a la propuesta hecha, formando una idea de proyecto y estructurando las mecanicas y/o codigos a implementar, tomando como eje principal que el usuario pueda ingresar el numero que desee de patrones a generar.
 \vspace{1cm}

Al tener las soluciones planteadas, buscamos e investigamos todos los conceptos que vamos a necesitar para el montaje del circuito en tinkercar, al igual que entender el funcionamiento de todos sus componentes para tener conciencia de como lo vamos a utilizar.






\vspace{14cm}

\section{Desarrollo} \label{contenido}
\subsection{Esquema de desarrollo}

Buscamos la solucion de programacion en c++, implementando las ideas pensadasy realizando un funcionamiento optimo, cumpliendo las problematicas propuestas.
 \vspace{1cm}
La solucion propuesta es hacer que el usuario sea quien incorpore el patron que desea mostrar en los leds, ingresando cada valor uno por uno, el cual le dará libertad de crear cualquier figura que desee.


\space

 -------------------------------------------------------------------
*codigo de c++*


Luego pensamos el montaje del arduino y todos sus componentes en el cual por medio de un transistor que estara configurado como un switch haciendo que los leds se enciendan o apaguen, dependiendo de lo ingresado por el usuario, estipulado por una determinada señal, con el objetivo de mostrar el patron ingresado.
 \vspace{1cm}
Por consecuente se tiene el codigo funcionando en c++, empezamos a plantear la contruccion del sistema en Tinkercar, cambiando el codigo en lenguaje de c++ al lenguaje de programacion que se maneja en Tinkercar.
 \vspace{1cm}

*Imagen del arduino*






\vspace{8cm}

\subsection{Algoritmo implementado}

blablabla

 \vspace{1cm}
 
blablabla

 \vspace{1cm}
 
blablabla

 \vspace{1cm}
 
blablalba 

 \vspace{1cm}



\subsection{Problemas que se presentaron}


Al empezar el montaje del sistema en tinkercar, hubo muchos errores en la conexion de los componentes, obligandonos a recrear en varias ocasiones la reestructuracion del sistema.
 \vspace{1cm}
 
 Teniendo el codigo en c++ 
 
 Integrado
 
 Datos puerto serial, por el integrado, se hace la interpretacion binaria de... integrado se encarga de encender o apagar el leds
Integrado manipula la cadena a traves de pulsos, diciendo cuales se encienden y apagan



\vspace{5cm}

\subsection{Solucion a problematicas presentadas}

blablabla 

\vspace{1cm}

blablabla

\vspace{8cm}
\section{Conclusión} \label{conclulsion}
blablalbalba

\vspace{1cm}


\end{document}
